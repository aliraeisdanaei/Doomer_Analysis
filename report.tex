\documentclass{article}
\usepackage[]{graphicx}
\usepackage[a4paper, total={6in, 10in}]{geometry}

\begin{document}

\pagenumbering{arabic}

\title{Analysis of Youth Emotional Loneliness}
\author{Ali Raeisdanaei}
\maketitle

\section{Introduction}
\subsection{Youth Emotional Loneliness is a problem}
Our young people are the loneliest. This has been proven by multiple sources [CITATIONS NEEDED]. 
This is true in the eastern countries as well as western ones [CITATION NEEDED].
In particular, Statistics Canada has identified those aged 15-24 as the loneliest demographic in Canada. 

With the increase of social media, availability of socialising, and ubandancy of information to young people especially, this is an alarming realisation. 
There seems to be a notion that something has changed in our collective culture. 
Exactly what has changed, and to what degree these changes are causing youth emotional loneliness is not known. 

\subsubsection{The Doomer as the archetype of a generation}

Recently, many cultural archetypes have emerged to describe the modern world. 
These archetypes are not mere creations for comical purposes. They pose real and deep messages through their lore as much as the legends and archetypes of antiquity.
Each character defines an aspect of the young person's existence and his interaction with the external world. 

Of these archetypes, often referred to as the \textit{wojack}, the doomer is the one that defines his generation.
The doomer, and by extension doomer-girl, is someone who is doomed to live a meaningless life. 
He is hopeless, dark, and alone. His world is that of the dilapidated post-soviet blocks. He sees no sunlight, no tree, no love. 

In comparison with the boomer, representing the naive and innocence of the baby boomers, the doomer has no prospects. 
The world the boomer created was to his benefit, and all but the problems, of social inequality, pollution, global warming, and all, are left for the doomer. 

Moreover, the doomer feels unquenchable nihilism. His pessimism is inspired by the likes of Schopenhauer. The works of existential thinkers of past generations is reimagined in his new modern world. 
Compared to the generations who endured the hardships of massacres throughout Europe, Asia, and Africa during the 19\textsuperscript{th} and 20\textsuperscript{th} centuries, 
this generation, at least the western generation, has seen relative peace. 
The Berlin Wall has collapsed. The Cold War has melted away. Social media has allowed for easier communication. 

So, why is the doomer doomed? Why is he the loneliest generation? And how is what he is experiencing something new?
\begin{figure}[h!]
    \includegraphics[width=\linewidth]{{Images/doomer1.jpg}}
    \caption[]{The Doomer}
    \label{fig:Doomer}
\end{figure}




    
\end{document}