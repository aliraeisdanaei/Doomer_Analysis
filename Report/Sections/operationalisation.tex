\documentclass[../report.tex]{subfiles}
\graphicspath{{\subfix{../Images/}}}

\addbibresource{../references.bib}

\begin{document}

\subsection{On the Shoulder of Giants}
This subsection will peruse the existing work to solidify how this project builds on literature in a novel way. 


\subsection{The Doomer Subreddit as a Study Ground}

The previous studies that sought to determine the causes of loneliness interviewed its subjects.
In particular, Fardghassemi \textit{et al.} interviewed young adults in London's deprived boroughs to answer such a question [CITATION NEEDED].  
This is approach is great, in so far as it asks the right people the right questions, yet it is limited in its vastness. 
The cost to find and conduct interviews is high; therefore, only a few participants can be used.
Fardghasssemi used only 48 participants. 

With any internet subculture, there is a rich amount of conversation available on forums, particularly, its subreddit. 
It can even be argued that many cultural elements of the internet, such as the doomer, are birthed on this platform.
Therefore, 
\href{www.reddit.com/r/doomer}{/r/rdoomer},
provides the researcher with more abundance of material to analyse.

On this subforum, there are personal accounts of loneliness, that stretch far into the time when the doomer was conceptualised. 
These personal accounts are to the same quality as the interviews produced by Fardghassemi \textit{et al.}

Thus, there are simply more information available to analyse for a lower cost. To analyse the high quality posts concerning the scope of the research under this forum, can give substantial results.
Causes may be discovered that were missed in the narrow number of participants in previous research led by traditional interviews.
Moreover, the information are made public by the users, the data scraped is not shared, and there is little to no invasiveness; therefore, there are less ethical considerations to be had.

\subsection{Research Question}

What are the causes of youth emotional loneliness amongst the doomers in reddit's
\href{www.reddit.com/r/doomer}{/r/rdoomer}?

\subsection{Operationalisation}

Doomers are then a good abstraction and representation of young people experiencing youth emotional loneliness. 
That is, given a phenomenon, it is wise to analyse its cultural elements and lore for the root causes. 

\end{document}