\documentclass[../report.tex]{subfiles}

\addbibresource{../references.bib}

\begin{document}

\subsection{On the Shoulder of Giants}
This subsection will peruse the existing work to solidify how this project builds on literature in a novel way. 

Karel Němeček's work is a theoretical analysis of memes and Doomers as social and cultural icons \cite{memes_reservoir}.
This project wishes to quantify the themes and causes in a computational method through textual analysis.

To study the causes of loneliness, Fardghassemi interviewed a group of 48 young adults from London's most deprived boroughs. The interviews were manually analysed for themes and causes of youth loneliness \cite{fardghassemi_interviews}.
This work wishes to compete with the interview style analysis of young adults at a higher scale.
Scraping the internet, particularly the Reddit's subforum,
\href{www.reddit.com/r/doomer}{/r/rdoomer},
can produce a higher number of personal accounts of substantial length at a lower cost.

Guntuku has a similar approach, in which she uses NLP to analyse the themes of twitter users posting about loneliness \cite{twitter_loneliness}.
This work wishes to use her approach to answer a cultural question about the novelty of Doomerism.

\subsubsection{Where I Stand and its Significance}
There is a problem with the youth becoming more lonely, and depressed. 
The issue so far has been analysed through a simplistic, traditional view of what the problems may be.
Doomerism, has arisen as a cultural form of the problem of the youth. 
To understand what it is, and if it is different from previously known phenomenon, interventions can be better created. 

\end{document}