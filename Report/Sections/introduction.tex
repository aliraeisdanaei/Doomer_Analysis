\documentclass[../report.tex]{subfiles}
\graphicspath{{../Images}}
\addbibresource{../references.bib}

\begin{document}

\subsection{The Doomer as the archetype of a generation}

Recently, many cultural archetypes have emerged, on the internet, to describe the modern generation and its world. 
These archetypes are not mere creations for comical purposes.
They pose real and deep messages through their lore as much as the legends and archetypes of antiquity.
Each character defines an aspect of the young person's existence and his interaction with the external world. 

Of these archetypes, often referred to as the \textit{Wojack}, the Doomer is the one that defines this generation of young people aged about 15 to 25.
The Doomer, and by extension Doomer-girl, is someone who is doomed to live a meaningless life.
This is the notion of Doomerism.
The Doomer is hopeless, dark, and alone.
His world is that of the dilapidated post-soviet blocks.
He sees no sunlight, no tree, no love. 

The use of iconography and specifically the Doomer as a cultural and social icon is studied by Karel Němeček \cite{memes_reservoir}. Karel Němeček identifies powerlessness and meaninglessness as the predominant forces of Doomerism. 

The Doomer is left with the problems, of social inequality, pollution, global warming, and everything else.
He feels powerless against the world.
Moreover, the Doomer feels unquenchable nihilism. His pessimism is inspired by the likes of Schopenhauer.
The works of existential thinkers of past generations is reimagined in his new world. 

% Compared to the generations who endured the hardships, massacres throughout Europe, Asia, and Africa during the 19\textsuperscript{th} and 20\textsuperscript{th} centuries, 
% this generation, at least the western generation, has seen relative peace. 
% The Berlin Wall has collapsed.
% The Cold War has melted away.
% Nutrition is readily available, at least in the west. 
% Meat consumption is at its highest \cite{meat_cons} \cite{meat_cons_canada} \cite{meat_cons_2022}.
% Social media is allowing for easier communication.
% The best technology is free to use and contribute to.
% Information, especially first-hand news and journalism, is a given.
% Everything seems to be, for the most part, \textit{better}.

% So, why is the Doomer doomed? Why is he the loneliest generation? And how is what he is experiencing something new?

\subsubsection{Youth Emotional Loneliness}

Depression and loneliness are not new to this generation, yet the relatability and virality of the Doomer icon seems to suggest that something has changed. This \textit{something} may be in our collective culture, the external world, prospects, relationships, and so on.

Young people are the loneliest demographic ever. This has been proven by multiple sources \cite{victor_yang_2012} \cite{bbc_radio4_2018} \cite{cigna_2018} \cite{cigna_newsroom_2021} \cite{hawkley_buecker_kaiser_luhmann_2020} \cite{von_soest_luhmann_gerstorf_2020} \cite{yougov_lonely_2019} \cite{stats_can_2021}.
In particular, Statistics Canada has identified those aged 15-24 as the loneliest demographic in Canada \cite{stats_can_2021}.

\begin{figure}[H]
    \centering
    \includegraphics[width=\linewidth]{{Doomer1.jpg}}
    % \includegraphics[width=1in]{{../Images/Doomer1.jpg}}
    \caption[]{The Doomer}
    \label{fig:Doomer}
\end{figure}

\subsubsection{Significance}
There exists a very dark icon that is gaining relevancy and become viral on the internet. 
Young people are identifying more and more with this character. 
Young people are also the loneliest demographic world-wide.
By not understanding what the Doomer is, we are potentially ignoring a major problem amongst young people. 

\subsection{Research Question}
Thus, it is important to understand what Doomerism is.
This is by itself is rather a trivial question; one can understand the elements of Doomerism by viewing a small subsample of its uses as icons. 
% This project aims to quantify these elements through textual analysis.
% That is, what are the exact composition of its themes and elements? 

This project aims to understand where Doomerism fits in with loneliness and depression.
How is the composition of themes of Doomerism different from that of loneliness and depression in general?
Is Doomerism a novel cultural archetype of a different era, or is it a new expression for long-existing phenomenons?

\subsection{On the Shoulder of Giants}
This subsection will peruse the existing work to solidify how this project builds on literature in a novel way. 

Karel Němeček's work is a theoretical analysis of memes and Doomers as social and cultural icons \cite{memes_reservoir}.
This project wishes to quantify the themes and causes in a computational method through textual analysis.

To study the causes of loneliness, Fardghassemi interviewed a group of 48 young adults from London's most deprived boroughs. The interviews were manually analysed for themes and causes of youth loneliness \cite{fardghassemi_interviews}.
This work wishes to compete with the interview style analysis of young adults at a higher scale.
Scraping the internet, particularly the Reddit's subforum,
\href{www.reddit.com/r/Doomer}{/r/rDoomer},
can produce a higher number of personal accounts of substantial length at a lower cost.

Guntuku has a similar approach, in which she uses NLP to analyse the themes of twitter users posting about loneliness \cite{twitter_loneliness}.
This work wishes to use her approach to answer a cultural question about the novelty of Doomerism.

Antie Andy has also studied loneliness on Reddit \cite{andy_2021}.
Andy studies the words used by users in loneliness forums as compared to other forums to identity themes amongst the users.
The approach here is similar, yet the scope of study is not just restricted to loneliness. 

\subsubsection{The Delta}
% \subsubsection{Where I Stand and its Significance}
% There is a problem with the youth becoming more lonely, and depressed. 
% The issue so far has been analysed through a simplistic, traditional view of what the problems may be.
% Doomerism, has arisen as a cultural form of the problem of the youth. 
% To understand what it is, and if it is different from previously known phenomenon, interventions can be better created. 
The approaches of cultural and psychological studies mentioned will be combined with textual analysis and computational techniques. 
The aim is to understand a psychological question through analysis of a cultural and social product.
\end{document}