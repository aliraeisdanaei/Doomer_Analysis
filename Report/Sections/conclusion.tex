\documentclass[../report.tex]{subfiles}

\addbibresource{../references.bib}

\begin{document}

There were four pieces of evidence to determine what Doomerism is in comparison with depression and loneliness. 
These evidences were based on the analysis done on posts from four Subreddits, /r/Doomer, /r/lonely, /r/loneliness, and /r/depression.

Firstly, the users of all /r/Doomer were unique, but so were the other Subreddits to each other. 

Secondly, the manual inspection of a sample of the posts from the four Subbreddits suggested that the Doomers were more nihilistic and concerned with the collapse of society, \textit{the doom}. Also, the users of the depression forum, posted more medical terms.
The Doomers are overwhelmingly and identifiably dark in comparison with the other categories.
% If anything this method gave results comparable to the interviews done by \cite{fardghassemi_interviews}.

Thirdly, the frequency analysis determined the exact themes that compose each of these social and mental health issues.
Its results correlated with that of the manual inspection, yet they did not find any other significant differences in the overall themes. 
Also, the significance of the \textit{doom} was not as large as suspected.
Similarly, the use of drugs as a theme was also comparably low across all the categories. 

Fourthly, the individual posts of each Subreddit were very divergent, especially that of the Doomers.
The Doomers are very similar to the users of /r/depression. 

\subsection{Discussion}
% Youth loneliness is a big problem, and we have seen a cultural icon bourne out of this phenomenon, the Doomer, figure \ref{fig:Doomer}. 
% To tackle youth loneliness, one has to understand one of its archetypes as accurately as possible. 
% Although, the age difference between all  the categories was negligible, the dominance of this icon in the public culture shows a significant trend, young people turning towards Doomerism.

% The results of this project show a clear diversity of themes and problems of the Doomers.
% As a whole, the icon is obsessed with nihilistic darkness and the approaching collapse of society and the world. 
% However, not all Doomers, those who actually post about their problems, can be classified as a Doomer.
% Many Doomers are just depressed and lonely.

% This is a significant result both culturally, socially, and for mental health. 
% The practitioner and the policy-maker that understands the exact problems of young people can better intervene.
% They know that nihilism and not any other issues in addition to those faced by previously studied groups are becoming more significant.

Given the virality and reaches of the doomer icon in internet culture, Doomerism is an emerging mental health and social problem plaguing the youth. 
It is important to know what it is, and understand its place within well studied problems. 

\subsection{Limitations and Threats to Validity}

% The first threat of the results arise from the bias that the author may have when studying this field. 
% This bias could have affected the manual inspection of the posts, and thereby affected the themes identified and studied. 

Only Reddit and these Subreddits were studied. 
The external validity of this approach is questionable; however, using Reddit was an appropriate choice for studying a cultural icon from the internet relating to the youth. 
The users had almost no diversity; the female user base was very small, and other identities almost non-existent.
Thus, part of the result of understanding Doomerism, concludes that it is mostly a male issue. 

The keyword selection is a naive approach to compose the themes of a problem.
The term "alcohol" could have had a different connotation and context in the post than what was counted.
Given the size of the corpus however, this effect should be small and evenly spread across the categories compared.
The frequency analysis lost a lot of the semantics and sentiments of the posts. 
The posts are much more identifiable given a better process.
Although the keywords were motivated by the manual inspection, many of them could have encoded assumptions and biases that have skewed the results.

No steps were taken to filter out for the quality of the posts. 
The posts from Reddit have a vote rating that could have been used to give different weight to themes expressed by the community.

Of course, this has many confounding complications, algorithmic and human.
The guidelines, rules, and administration of Subreddits are notoriously brutal, and they could have skewed the data in a major way.

The use of bots in these forums were not a huge problem. 
These forums are more serious, and posts have already been filtered for a significant enough number of words.

\subsection{Further Steps}
The data spanned across the COVID-19 pandemic. 
Therefore, a longitudinal analysis could be useful for future work to determine the changes of particular themes and causes due to the global effects of the pandemic.

More sophisticated natural language processing tools can be used to more accurately measure the composition of the themes for analysis. 
As well, the classification algorithm could be improved upon by more sophisticated models and features of the data. 

\subsection{Ethics}
The issue of ethics was closely guarded for this research.
The people in consideration for this research are vulnerable groups; therefore no data created for this research has been shared. 
Moreover, all notes created for the analysis of these posts have also been made private. 
For the most part, identifiable features of the posts were hidden and not discussed.

All the data, that was used, are made public by the users themselves.
The users are sharing these posts for others to read and comment on; thus, in a way, giving informed consent.
Free software and widely available scrapers were used to retrieve that data. 
Any person could have access to the posts made by these users. 

\subsection{Acknowledgments}
I thank my great friend Ali Samani, the psychology major turned data scientist, for giving me great pointers for my data science work.

I thank Lina Marsso for giving me great advice and showing me where to look for things. 

Furthermore, I also thank Lina's great friend Victoria Oldemburgo de Mello for pointing me to some related research that really helped. 

\subsection{Software Used}
All the software used for this project is available here, and is free to other researchers 
\url{github.com/aliraeisdanaei/Doomer_Analysis}.
\end{document}