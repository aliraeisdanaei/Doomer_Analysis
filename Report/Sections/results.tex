\documentclass[../report.tex]{subfiles}
\graphicspath{{../Images}}
\addbibresource{../references.bib}

\begin{document}

\subsection{The Commonality of the users}
The users of these Subreddits were generally unique.
That is, only a few Doomers posted in the other similar forums. 
This is our first evidence for the Doomers being a distinct identity. 

The number of unique users of each of the Subreddits were greater than a thousand; however, the common users between the Doomers and the other Subreddits were very low. 
There were only 13, 6, 14 common users between /r/Doomer, /r/lonely, /r/loneliness, and /r/depression respectively. 

The greatest number of common users were actually not between /r/lonely and /r/loneliness, 31: this, I believe, is to the smaller user base of /r/loneliness. 
The greatest commonality was between /r/depression and /r/lonely, 161, suggesting that lonely people are generally depressed and vice-versa.

There were no common users between all the Subreddits, and only 2 and 1 common users between /r/Doomer, /r/depression, /r/lonely, and /r/loneliness.

\subsection{Manual Inspection}
A sample of these posts were read to identify the themes and causes present in the categories.
A sample of 50 posts from both /r/Doomer and /r/loneliness were read. 20 posts from /r/lonely and /r/depression were read. 

\subsubsection{Themes of /r/Doomer}
Doomers are predominantly male, young, and single. 
There are lots of mentions of school and a lot of accounts of their actual age and gender. 
In Doomer terms, they are plagued by \textit{no gf}. 
They have seldom talked to girls, they are worried about their virginity, and they feel utterly helpless. 
There are lots of mentions of relationship issues and porn addictions. 

The largest theme that arose from the sample reading were themes of \textbf{nihilism} and \textbf{doom}. 
One can argue that these are not separate entities; nihilism seems to be a subset of doom, and doom seems to be a subset of nihilism.
Yet, doom encompasses more than just nihilism.
Doom is the notion that the world is hopelessly going to be extinct. 
Our civilisation is bound to collapse, and the Doomers see it clearly.
The world is doomed, and the Doomers are the outcasts who are first experiencing this. 
Thus, their nihilism can be seen to arise from this doom. 
There was a simple poetry to the doom of the Doomer; a sample of this is given in section \ref{appendix}.

The second-biggest theme that arose from the manual inspection was \textbf{loneliness} and \textbf{\textit{no gf}}. 
I will be using this term to refer to being single (having no girlfriend).
This term is not only an abbreviation, it's an esoteric term that signifies the obsessiveness of the Doomers towards their being single.
This obsession is somewhat close to the ideas of \textit{incels} (involuntary celibates), a truly toxic subculture arising from the internet. 
These Doomers really blame not having sex as an evidence of them being doomed. 

There are two comical tropes that explain this. 
The first is the Doomer finding the Doomer girl, the exact female counterpart of the Doomer, and them being \textit{doomed} together. 
The sample of posts inspected did have females, but they are rare. 
It is an elusive dream or a joke within this community of ever finding the Doomer girl. 
In fact of the top posts of /r/reddit, there is one of a Doomer couple holding hands. 

The second feature that displays the comedy and obsession of the Doomers with girls and sex is the idea of a state mandated girlfriend. 
The Doomer is so doomed, that the only plausible intervention by the government issues girlfriends. 
This second example is a stretch, as it did not come during the inspection, and it may be more attributed the afformentioned incel subculture. 

Other causes include \textbf{suicide}, \textbf{a bad childhood}, \textbf{a dead end job}, and to lesser extents issues of \textbf{mental health} and \textbf{drugs}. The two latter causes were less relevant to the sample Doomer posts inspected.

\subsubsection{Themes of /r/lonely and /r/loneliness}
The sample posts inspected between the two Subreddits were identical in themes. 


The lonely uses are an older group.
They are more likely to be female, in a relationship, or even married.
In the 50 posts in the Doomer set, there was not a single post mentioning they were in or ever were in a marriage. 
The Doomer looks up to marriage as perhaps the naive, traditional embodiment of the good life, the \textit{undoomed} life as it were. 

The lonely person seems to be dealing with issues that are simpler and more natural to the human condition. 
Friends are growing distant as the person ages, they cannot find friends, or a romantic partner, they have gone through a break up, and so on. 
Therefore, the first cause of lonelinessim identified were \textbf{relationship issues}.

The other themes identified were issues of \textbf{social issues}. The users mentioned their being too shy and introverted to change anything. 
There were also mentions of \textbf{unlovability}.
The users could have been in marriages, yet they felt emotionally alone.
Their souls and beings were not cared for as they would wish.

The loner posts had more mentions of hope and lights at the end of the tunnel. 
The loner has some sense of optimism for how they can, or in the least, should change things.
The Doomer's identity comprises a deep, dark core of nihilism and doom.
The Doomer is simply not lonely, they are \textit{doomed} to be lonely and unloved. 

\subsubsection{Themes of /r/depression}
The users of this Subreddit were far \textbf{medical} than any of the other users. 
In fact, they seemed to be dealing with these issues but still having some problems. 
They mentioned medications or therapy, and ways they are trying to remedy their diagnosed problems. 

Many of the posts mentioned \textbf{issues and responsibilities} that they were dealing with.
A breakup or a death in the family was very commonly mentioned. 

There was also an \textbf{emotional emptiness}. Many users described their emotions in terms of being sad, having or lacking in pain, energy, or motivation. 
The theme of \textbf{suicide} was mentioned here as well but not to the extent of /r/Doomer.

\subsubsection{Conclusion of Manual Inspection}
Although, all the problems the users of these Subreddits were dealing were serious and need of attention, there were degrees to their \textit{troubles}.

The Doomer is, by far, the most dark and nihilistic of the other groups. In this regard he is distinct. 

\subsection{Frequency Analysis of Themes}
The posts were analysed for their frequencies; the word-clouds produced by each of the Subreddits can be seen in section \ref{appendix}.

Motivated by the themes identified in the manual inspection, a list of keywords that would identify these themes were compiled. 
They can be seen in table \ref{tab:keywords}.
These words were then searched through the corpus of posts by using regular expressions.

The number of words found from a given them was then expressed as a percentage of all the keywords matched.
In such a way each category received its composition as a percentage of causes, the results of which can be seen in table \ref{tab:composition}.
Other visualisations of these compositions can also be seen in section \ref{appendix}.

The results of these compositions support the findings of the manual inspection. 
The Doomers are much more nihilistic and the users of /r/depression more medical.
The theme of doom was only seen, unsurprisingly, by the Doomer; although, this was still a low percentage of composition for the Doomer. 
The Doomer is more nihilistic than he is obsessed with the \textit{doom}.

The prevelance of drugs is much less than suggested by \cite*{twitter_loneliness}, only being high as 2.25\%.
The two loneliness forums were similar in compositions as suggested by the manual inspection. 
Both have much higher levels of loneliness as compared to the other categories. 

In all other themes, the categories were very similar. 

\subfile{table_frequency}

\subsection{Classification of Individual Posts}
The composition of the categories was used as a vector. 
The same approach of finding the prevalence of keywords by themes were used on individual posts.
This prevalence was also used as a vector to measure the similarity of the posts to the average posts of each category. 

The similarity function used was a modified version of the euclidean norm: in this function, the components of the vectors that were different across the categories had a higher degree. 
The modification to the euclidean norm was done to increase the accuracy of the results.

Then a post would be blindly classified to the category in which its composition was most similar. 
The results of this classification can be seen in table \ref{tab:pred_results}.
This approach shows us the individual deviation of the posts and if the Doomer can be correctly classified individually against the other categories.

The results of this show that individual posts of Doomers are mostly similar to the users of /r/depression. 
It also concludes that Doomer's are a diverse set of individuals who are both lonely and depressed; however, there is a group, albeit small, that exhibits more nihilism and doom than the other categories.
There was an equal similarity to the composition of Doomerism across the Doomer's posts to the loneliness forums.

Interestingly, the Doomers were more similar to the users of the depression forum, than the users of the depression forum to the Doomers. 

To check the similarity of the two loneliness forums, the classifier checked the similarity of individual posts in each category to the other. 
Its results can be seen in table \ref{tab:loneliness_pred_results}.
Both of the categories were almost identical to each other. 
The model was not better at guessing them over a random classifier.

A binary classification was also performed to test if the categories were distinct from the others.
The composition vectors of the other forums were averaged as being the \textit{other}.
For the two loneliness forums, the two composition vectors were grouped together, as motivated by table \ref{tab:loneliness_pred_results}.
The results of the binary classification can be seen in table \ref{tab:binary_pred_results}.
The binary classification was not very accurate for any of the categories, suggesting that all were very similar to each other. 
This is expected, as the forums and the users concern themselves with similar issues of mental health.

However, if Doomerism was expressed by any of the other categories, a mixture of loneliness and depression, the results would have shown a 50\% accuracy. 
The model predicted it slightly better than random, showing that Doomerism is not juts loneliness and depression.
It is something distinct, but its distinction is to the others as loneliness and depression are to the others\-- it is a small distinction.


\subfile{table_predictions}
	
\end{document}