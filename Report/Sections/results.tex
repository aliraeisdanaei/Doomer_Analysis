\documentclass[../report.tex]{subfiles}

\addbibresource{../references.bib}

\begin{document}

\subsection{The Commonality of the users}
The users of these Subreddits were generally unique.
That is, only a few doomers posted in the other similar forums. 
This is our first evidence for the doomers being a distinct identity. 

The number of unique users of each of the Subreddits were greater than a thousand; however, the common users between the doomers and the other Subreddits were very low. 
There were only 13, 6, 14 common users between /r/doomer, /r/lonely, /r/loneliness, and /r/depression respectively. 

The greatest number of common users were actually not between /r/lonely and /r/loneliness, 31: this, I believe, is to the smaller user base of /r/loneliness. 
The greatest commonality was between /r/depression and /r/lonely, 161, suggesting that lonely people are generally depressed and vice-versa.

There were no common users between all the Subreddits, and only 2 and 1 common users between /r/doomer, /r/depression, /r/lonely, and /r/loneliness.

\subsection{Manual Inspection}
A sample of these posts were read to identify the themes and causes present in the categories.
A sample of 50 posts from both /r/doomer and /r/loneliness were read. 20 posts from /r/lonely and /r/depression were read. 

\subsubsection{Themes of /r/doomer}
Doomers are predominantly male, young, and single. 
There are lots of mentions of school and a lot of accounts of their actual age and gender. 
In doomer terms, they are plagued by \textit{no gf}. 
They have seldom talked to girls, they are worried about their virginity, and they feel utterly helpless. 
There are lots of mentions of relationship issues and porn addictions. 

The largest theme that arose from the sample reading were themes of \textbf{nihilism} and \textbf{doom}. 
One can argue that these are not separate entities; nihilism seems to be a subset of doom, and doom seems to be a subset of nihilism.
Yet, doom encompasses more than just nihilism.
Doom is the notion that the world is hopelessly going to be extinct. 
Our civilisation is bound to collapse, and the doomers see it clearly.
The world is doomed, and the doomers are the outcasts who are first experiencing this. 
Thus, their nihilism can be seen to arise from this doom. 

There was a simple poetry to the doom of the doomer. Here is a sample written by doomers.

\begin{quotation}
    The prophetic vision of the holy executioner finally begins to take shape and unfold
    Nothing remains to view the final breaths of a dying star many eons after all signs
    of life ceased their existence on the surface of our beautiful planet
    leaving behind a still silhouette of memory of a once magnificently brilliant world
    The prophetic vision of the holy executioner finally begins to take shape and unfold where the monochrome deafening silence echoes throughout the remnants of what was our beautiful world.
\end{quotation}

% \begin{multicols}{2}
\begin{quotation}
	\noindent
	An exiled hermit\\
	These desolate plains I roam\\
	From the cosmic infancy\\
	I float like a lost comet\\
	Drifting through empty space\\
	Deprived of self\\
	I find solace within painful memory\\
	Of my sadistic lover,\\
	Life\\
	The ghost of ecstasy\\
	Lurking in the sorrows\\
	Which belong to me alone\\
	To serve as the gate to the firmament\\
	And contain the flames of agony\\
	In which I've forged the poisoned knife\\
	With which I'll put the suicidal god\\
	Out of its eternal misery\\
	Before abandoning the heavens\\
	Consumed by the flames of my passion\\
	Casting a shadow to eclipse the sun\\
	Burning brighter than the morning star\\
	At the edges of the dawn\\
	To break free from the chains of blind devotion\\
	devouring the minds of man\\
	That shall signal the inevitable collapse of tyranny\\
	Of paradise and hell\\
	Born out our asymmetry\\
	In ascendancy\\
\end{quotation}
% \end{multicols}

The second biggest theme that arose from the manual inspection was \textbf{loneliness} and \textbf{\textit{no gf}}. 
I will be using this term to refer to being single (having no girlfriend).
This term is not only an abbreviation, it's an esoteric term that signifies the obsessiveness of the doomers towards their being single.
This obsession is somewhat close to the ideas of \textit{incels} (involuntary celibates), a truly toxic subculture arising from the internet. 
These doomers really blame not having sex as an evidence of them being doomed. 

There are two comical tropes that explain this. 
The first is the doomer finding the doomer girl, the exact female counterpart of the doomer, and them being \textit{doomed} together. 
The sample of posts inspected did have females, but they are rare. 
It is an elusive dream or a joke within this community of ever finding the doomer girl. 
In fact of the top posts of /r/reddit, there is one of a doomer couple holding hands. 

The second feature that displays the comedy and obsession of the doomers with girls and sex is the idea of a state mandated girlfriend. 
The doomer is so doomed, that the only plausible intervention by the government issues girlfriends. 
This second example is a stretch, as it did not come during the inspection, and it may be more attributed the afformentioned incel subculture. 

Other causes include \textbf{suicide}, \textbf{a bad childhood}, \textbf{a dead end job}, and to lesser extents issues of \textbf{mental health} and \textbf{drugs}. The two latter causes were less relevant to the sample doomer posts inspected.

\subsubsection{Themes of /r/lonely and /r/loneliness}
The sample posts inspected between the two Subreddits were identical in themes. 


The lonely uses are an older group.
They are more likely to be female, in a relationship, or even married.
In the 50 posts in the doomer set, there was not a single post mentioning they were in or ever were in a marriage. 
The doomer looks up to marriage as perhaps the naive, traditional embodiment of the good life, the \textit{undoomed} life as it were. 

The lonely person seems to be dealing with issues that are simpler and more natural to the human condition. 
Friends are growing distant as the person ages, they cannot find friends, or a romantic partner, they have gone through a break up, and so on. 
Therefore, the first cause of lonelinessim identified were \textbf{relationship issues}.

The other themes identified were issues of \textbf{social issues}. The users mentioned their being too shy and introverted to change anything. 
There were also mentions of \textbf{unlovability}.
The users could have been in marriages, yet they felt emotionally alone.
Their souls and beings were not cared for as they would wish.

The loner posts had more mentions of hope and lights at the end of the tunnel. 
The loner has some sense of optimism for how they can, or in the least, should change things.
The doomer's identity comprises a deep, dark core of nihilism and doom.
The doomer is simply not lonely, they are \textit{doomed} to be lonely and unloved. 

\subsubsection{Themes of /r/depression}
The users of this Subreddit were far \textbf{medical} than any of the other users. 
In fact, they seemed to be dealing with these issues but still having some problems. 
They mentioned medications or therapy, and ways they are trying to remedy their diagnosed problems. 

Many of the posts mentioned \textbf{issues and responsibilities} that they were dealing with.
A breakup or a death in the family was very commonly mentioned. 

There was also an \textbf{emotional emptiness}. Many users described their emotions in terms of being sad, having or lacking in pain, energy, or motivation. 
The theme of \textbf{suicide} was mentioned here as well but not to the extent of /r/doomer.

\subsubsection{Conclusion of Manual Inspection}
Although, all the problems the users of these Subreddits were dealing were serious and need of attention, there were degrees to their \textit{troubles}.

The doomer is, by far, the most dark and nihilistic of the other groups. In this regard he is distinct. 

\subsection{Frequency Analysis of Themes}

\begin{table}[H]
		\centering
		\scalebox{0.8}{
		\begin{tabular}{l | c | c | c | c}
			\toprule
			themes & /r/doomer & /r/lonely & /r/loneliness & /r/depression \\
			\midrule
			nihilism &               18.90 &                 9.90 &                 9.59 &             13.07 \\
				doom &                2.73 &                 0.22 &                 0.13 &              0.30 \\
				alone &               17.48 &                31.13 &                33.60 &             15.03 \\
			medical &                5.97 &                 5.93 &                 5.74 &             12.39 \\
		relationship &               12.90 &                15.20 &                14.02 &             10.66 \\
			childhood &               10.76 &                 8.90 &                 9.40 &             12.13 \\
				job &                5.68 &                 4.42 &                 4.26 &              6.54 \\
		mental health &               11.25 &                 8.93 &                 8.68 &             16.88 \\
				drugs &                2.25 &                 0.76 &                 0.81 &              1.27 \\
		social issues &                7.10 &                11.15 &                10.30 &              6.20 \\
	responsibilities &                4.99 &                 3.47 &                 3.45 &              5.54 \\
			\bottomrule
		\end{tabular}
		}
		\caption{Composition of Themes by Percentage}
		\label{tab:composition}
\end{table}

\begin{figure}[H]
	\begin{tikzpicture}
		\scalebox{0.8}{
		\pie[hide number]{
			18.90/nihilism,
			2.73/doom,
			17.48/alone,
			5.97/medical,
			12.90/relationship,
			10.76/childhood,
			5.68/job,
			11.25/mental health,
			2.25/drugs,
			7.10/social issues,
			4.99/responsibilities
		}
		}
	\end{tikzpicture}
	\caption{Composition of the Doomer}
	\label{fig:comp_doomer}
\end{figure}
	
\end{document}