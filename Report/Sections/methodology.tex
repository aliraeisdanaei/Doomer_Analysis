\documentclass[../report.tex]{subfiles}
\addbibresource{../references.bib}

\begin{document}

\subsection{The Data}
The best place to study the Doomer is its subforum on Reddit, /r/Doomer. To compare the Doomer with other similar categories, the subforums of, /r/lonely, /r/loneliness, and /r/depression were also scraped.

All the data was filtered for posts that were of greater length than 500. The data is described in table \ref{tab:metadata}.

\subsubsection{The Users}
The users of the internet, Reddit, and specifically these Subreddits, skew towards young, English-speaking, males. 
Some posters include their age and gender information.
This is really rare, but to understand the users, this information was extracted using regular expressions.
The results of this can be seen in tables \ref{tab:ages} and \ref{tab:gender}.

\subsection{The Commonality of the users}
The users of these Subreddits were generally unique.
That is, only a few Doomers posted in the other forums. 
% This is our first evidence for the Doomers being a distinct identity. 

The number of unique users of each of the Subreddits were greater than a thousand; however, the common users between the Doomers and the other Subreddits were very low. 
There were only 13, 6, 14 common users between /r/Doomer, /r/lonely, /r/loneliness, and /r/depression respectively. 

The greatest number of common users were actually not between /r/lonely and /r/loneliness, 31. 
The greatest commonality was between /r/depression and /r/lonely, 161, suggesting that lonely people are generally depressed and vice-versa.

There were no common users between all the Subreddits.
Only 2 common users existed between /r/Doomer, /r/depression, /r/lonely, and Only 2 common users between /r/Doomer, /r/depression, /r/loneliness.

\subfile{table_metadata}

\subsection{Approach}
To start off, the posts were manually examined by a human to identify some causes and themes from the different Subreddits. 
This approach is similar to Fardghassemi's analysing of the interviews \cite{fardghassemi_interviews}.

When the causes have were identified from the manual approach, a list of terms was identified.
These terms will be analysed for their frequencies in a similar approach to the work of Guntuku \cite{twitter_loneliness}.

Thirdly, a composition of the causes are created for each category, as the measure of the frequency of the terms in the corpus of posts.
This data could be used to train a model to classify a Doomer from the other categories understanding the individual post diversity of the categories and their similarities.
% This would answer the research question of the Doomer having a distinct and novel identity.

\end{document}