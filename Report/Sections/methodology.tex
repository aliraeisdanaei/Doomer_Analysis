\documentclass[../report.tex]{subfiles}
\addbibresource{../references.bib}

\begin{document}

\subsection{The Data}
The best place to study the doomer is its subforum on Reddit, /r/doomer. To compare the doomer with other similar characteristics, the subforums of, /r/lonely, /r/loneliness, and /r/depression were also scraped.

All the data was filtered for posts that were of greater length than 500. The data is described in table \ref{tab:metadata}.
The users of Reddit, and specifically these Subreddits, skew towards young, English-speaking, males. 
\subfile{table_metadata}

\subsection{Approach}
To start off, the posts were manually examined by a human to identify some causes and themes from the different Subreddits. 
This approach is similar to Fardghassemi's analysing of the interviews \cite{fardghassemi_interviews}.

When the causes have then been identified from the manual approach, a list of terms will be identified.
These terms will be analysed for their frequencies in a similar approach to the work of Guntuku \cite{twitter_loneliness}.
If terms pertaining to one cause always have a higher frequency than another in one Subreddit over another, it is clear that cause is more prevalent. 

Thirdly, a composition of the causes are created for a doomer, lonely person, and depressed person, as the measure of the frequency of the terms in the corpus of posts. This data could be used to train a model to classify a doomer from the other categories. This would answer the research question of the doomer having a distinct and novel identity.

\end{document}