\documentclass{article}
\usepackage[]{graphicx}
\usepackage{hyperref}
\usepackage[a4paper, total={6in, 10in}]{geometry}

\begin{document}

\pagenumbering{arabic}

\title{Analysis of Youth Emotional Loneliness}
\author{Ali Raeisdanaei}
\maketitle

\section{Introduction}
\subsection{Youth Emotional Loneliness is a problem}
Our young people are the loneliest. This has been proven by multiple sources [CITATIONS NEEDED]. 
This is true in the eastern countries as well as western ones [CITATION NEEDED].
In particular, Statistics Canada has identified those aged 15-24 as the loneliest demographic in Canada. 

With social media increasing the availability of socialising, and abundance of information to young people, this is an alarming realisation. 
There seems to be a notion that something has changed in our collective culture. 
Exactly what has changed, and to what degree these changes are causing youth emotional loneliness is not known, and is the concern of this project.

\subsubsection{The Doomer as the archetype of a generation}

Recently, many cultural archetypes have emerged, predominantly on the internet, to describe the modern generation and its world. 
These archetypes are not mere creations for comical purposes. They pose real and deep messages through their lore as much as the legends and archetypes of antiquity.
Each character defines an aspect of the young person's existence and his interaction with the external world. 

Of these archetypes, often referred to as the \textit{wojack}, the doomer is the one that defines this generation of young people aged about 15 to 25.
The doomer, and by extension doomer-girl, is someone who is doomed to live a meaningless life. 
He is hopeless, dark, and alone. His world is that of the dilapidated post-soviet blocks. He sees no sunlight, no tree, no love. 

In comparison with the boomer, representing the naive and innocence of the baby boomers, the doomer has no prospects. 
The world the boomer created, was to his benefit, and all but the problems, of social inequality, pollution, global warming, and everything else, are left for the doomer to handle. 

Moreover, the doomer feels unquenchable nihilism. His pessimism is inspired by the likes of Schopenhauer. The works of existential thinkers of past generations is reimagined in his new world. 

Compared to the generations who endured the hardships, massacres throughout Europe, Asia, and Africa during the 19\textsuperscript{th} and 20\textsuperscript{th} centuries, 
this generation, at least the western generation, has seen relative peace. 
The Berlin Wall has collapsed.
The Cold War has melted away.
Nutrition is readily available [CITATIONS NEEDED]. 
Meat consumption is at its highest [CITATION NEEDED].
Social media is allowing for easier communication.
The best technology is free to use and contribute to.
Information, especially first-hand news and journalism, is a given.
Everything seems to be, for the most part, \textit{better}.

So, why is the doomer doomed? Why is he the loneliest generation? And how is what he is experiencing something new?
\begin{figure}[h!]
    \includegraphics[width=\textwidth]{{../Images/doomer1.jpg}}
    \caption[]{The Doomer}
    \label{fig:Doomer}
\end{figure}

\subsubsection{The Doomer Subreddit as a Study Ground}

The previous studies that sought to determine the causes of loneliness interviewed its subjects.
In particular, Fardghassemi \textit{et al.} interviewed young adults in London's deprived boroughs to answer such a question [CITATION NEEDED].  
This is approach is great, in so far as it asks the right people the right questions, yet it is limited in its vastness. 
The cost to find and conduct interviews is high; therefore, only a few participants can be used.
Fardghasssemi used only 48 participants. 

With any internet subculture, there is a rich amount of conversation available on forums, particularly, its subreddit. 
It can even be argued that many cultural elements of the internet, such as the doomer, are birthed on this platform.
Therefore, 
\href{www.reddit.com/r/doomer}{/r/rdoomer},
provides the researcher with more abundance of material to analyse.

On this subforum, there are personal accounts of loneliness, that stretch far into the time when the doomer was conceptualised. 
These personal accounts are to the same quality as the interviews produced by Fardghassemi \textit{et al.}

Thus, there are simply more information available to analyse for a lower cost. To analyse the high quality posts concerning the scope of the research under this forum, can give substantial results.
Causes may be discovered that were missed in the narrow number of participants in previous research led by traditional interviews.
Moreover, the information are made public by the users, the data scraped is not shared, and there is little to no invasiveness; therefore, there are less ethical considerations to be had.

\subsection{Research Question}

\subsection{Operationalisation}
\end{document}